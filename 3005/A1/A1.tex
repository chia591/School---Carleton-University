\documentclass[a4 paper]{article}
% Set target color model to RGB
\usepackage[inner=2.0cm,outer=2.0cm,top=2.5cm,bottom=2.5cm]{geometry}
\usepackage{setspace}
\usepackage[rgb]{xcolor}
\usepackage{verbatim}
\usepackage{subcaption}
\usepackage{amsgen,amsmath,amstext,amsbsy,amsopn,tikz,amssymb,tkz-linknodes}
\usepackage{fancyhdr}
\usepackage[colorlinks=true, urlcolor=blue,  linkcolor=blue, citecolor=blue]{hyperref}
\usepackage[colorinlistoftodos]{todonotes}
\usepackage{rotating}
%\usetikzlibrary{through,backgrounds}
\hypersetup{%
pdfauthor={Ashudeep Singh},%
pdftitle={Assignment 1},%
pdfkeywords={Tikz,latex,bootstrap,uncertaintes},%
pdfcreator={PDFLaTeX},%
pdfproducer={PDFLaTeX},%
}
%\usetikzlibrary{shadows}
% \usepackage[francais]{babel}
\usepackage{booktabs}
\input{macros.tex}


\begin{document}
\homework{Assignment \#1}{Due: Jan. 31, 2020 (11:59 PM)}{Ahmed El-Roby}{}{}{}
\textbf{Instructions}: Read all the instructions below carefully before you start working on the assignment, and before you make a submission.
\begin{itemize}
    \item The accepted formats for your submission are: pdf, docx, txt. More details below. 
    \item You can either write your solutions in the tex file (then build to pdf) or by writing your solution by hand or using your preferred editor (then convert to pdf or docx). However, you are encouraged to write your solutions in the tex file (5\% bonus). If you decide not to write your answer in tex, it is your responsibility to make sure you write your name and ID on the submission file.
    \item If you use the tex file, make sure you edit line 28 to add your name and ID. Only write your solution and do not change anything else in the tex file. If you do, you will be penalized.
    \item All questions in this assignment use the university schema discussed in class (on culearn), unless otherwise stated.
    \item For SQL questions, upload a text file with your queries in the format shown in the file ``template.txt'' uploaded on culearn. An example submission is in the file ``sample.txt''. You will be penalized if the format is incorrect or there is no text file submission. 
\end{itemize}


\problem{1:}{6}
Answer the following questions using the university schema discussed in class: 
\subproblem{a} The primary key for the \emph{advisor} relation is \emph{s\_id}. Suppose a student can have more than one suprvisors. Would \emph{s\_id} still be a primary key in \emph{advisor}? If yes, why? If not, what would be a suitable primary key?\indent (2 marks)

\vspace{3em}


\subproblem{b} The primary key for \emph{prereq} is both attributes \emph{course\_id} and \emph{prereq\_id}. Why wouldn't only \emph{course\_id} work as primary key?\indent (2 marks)\\

\vspace{3em}


\subproblem{c} Given the existing schema of \emph{teaches}, two or more instructors can teach the same section. How can the primary key be changed to restrict a section to one instructor only?\indent (2 marks)\\

\vspace{3em}

\problem{2:}{12}
Consider the following bank database schema:\\
\emph{branch(\underline{branch\_name}, branch\_city, assets)}\\
\emph{customer(\underline{ID}, customer\_name, customer\_street, customer\_city)}\\
\emph{loan(\underline{loan\_number}, branch\_name, amount)}\\
\emph{borrower(\underline{ID}, \underline{loan\_number})}\\
\emph{account(\underline{account\_number}, branch\_name, balance)}\\
\emph{depositor(\underline{ID}, \underline{account\_number})}

\noindent Write an expression in relational algebra to find the following:

\subproblem{a} Find the cities that host branches that have a loan that is greater than \$50000.\indent (3 marks)\\

\vspace{3em}


\subproblem{b} Find the ID of each depositor who has an account with a balance greater than \$50000 at the ``Nepean'' branch.\indent (3 marks)\\

\vspace{3em}

\subproblem{c} Find the names of customers who have at least one loan amount that is greater than at least one account balance.\indent (6 marks)\\

\vspace{3em}



\problem{3:}{33}
Using the university database schema discussed in class, write the SQL statements that do:
\subproblem{a} Create a new course (``Aces of Databases'') with ID (``COMP5118'') in the Computer Science department (``Comp. Sci.'') with 0 credit hours.\indent (3 marks)\\

\vspace{10em}

\subproblem{b} Create a section 'A' for this course in the Winter of 2020 with no known location or time, yet.\indent (4 marks)\\

\vspace{10em}

\subproblem{c} Enroll all students in the department into this course.\indent(5 marks)\\

\vspace{10em}

\subproblem{d} One student with ID 12345 cannot take this course because of violating the prerequisite requirements (didn't pass COMP3005). Unregister this student from the new section.\indent (3 marks)\\

\vspace{10em}

\subproblem{e} For each student who took a course at least twice, show the course ID and the student ID.\indent (5 marks)\\

\vspace{10em}


\subproblem{f} Find the ID and name of instructors who never gave a grade 'A' in the courses they taught (note that instructors who never taught a course satisfy this condition).\indent (5 marks)\\

\vspace{10em}


\subproblem{g} Rewrite the previous query so that you make sure that the instructor taught at least one course.\indent (5 marks)\\

\vspace{10em}

\subproblem{h} Find the lowest, across all departments, of the per-department maximum salary.\indent (3 marks)

\vspace{10em}


\problem{4}{5}
Consider the following car insurance schema:\\
\emph{person(\underline{driver\_id}, name, address)}\\
\emph{car(\underline{licence}, model, year)}\\
\emph{accident(\underline{report\_number}, date, location)}\\
\emph{owns(\underline{driver\_id}, \underline{licence\_plate})}\\
\emph{participated(\underline{report\_number}, \underline{licence\_plate}, driver\_id, damage\_amount)}\\
Write SQL queries to:\\
\subproblem{a} Find the number of accidents involving a car belonging to a person named ``Ahmed El-Roby''.\indent (3 marks)

\vspace{10em}


\subproblem{b} Update the damage amount for the car with licence plate ``DB007'' in the accident with report number ``AR2020'' to \$3000.\indent (2 marks)

\vspace{10em}

\end{document}

